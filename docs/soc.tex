\documentclass[12pt]{article}
\usepackage[slovak]{babel}
\usepackage{mathptmx}

\usepackage[a4paper,left=35mm,
                    right=25mm,
                    top=25mm,
                    bottom=25mm]{geometry}

\linespread{1.5}   % 1.5 riadkovanie


\begin{document}

%
% TODO:
% [x] Presný názov školy a jej presná adresa, mesto (tiež jeho poštové
%     smerovacie číslo).
% [x] Stredoškolská odborná činnosť.
% [?] Číslo a názov súťažného odboru, do ktorého autor prácu prihlasuje.
% [?] Názov práce (nie dlhý, má presne vystihovať obsah práce). 
% [x] Meno a priezvisko autora a všetkých spoluautorov.
% [x] Ročník štúdia.
% [x] Meno a priezvisko školiteľa so všetkými titulmi (ak má práca aj konzultanta
%     tak aj meno a priezvisko konzultanta so všetkými jeho titulmi).
% [ ] presný názov zriaďovateľa. 
% [x] Miesto a kalendárny rok dokončenia práce.
%
\begin{titlepage}
    \setlength{\parindent}{0pt}

    \begin{center}
        Gymnázium \\
        Veľká okružná 22, 010 01 Žilina

        \vspace{7cm}
        \Huge Botanica - simulátor rastlín na báze celulárnych automatónov

        \vspace{1.13cm}
        \Large Stredoškolská odborná činnosť

        \vspace{2.12cm}
        \normalsize Č. odboru: 11
    \end{center}

    \vfill

    \begin{minipage}{0.75\textwidth}
        Riešitelia: František Knapec, Michael Sklenka, Marek Beňo \par
        Ročník štúdia: 4.
    \end{minipage}
    \hfill
    \begin{minipage}{0.23\textwidth}
        \hfil % basically align right
        \begin{tabular}{rc}
            Mesto: & Žilina \\
            Rok:   & 2025
        \end{tabular}
    \end{minipage}
\end{titlepage}

\begin{titlepage}
    \setlength{\parindent}{0pt}

    \begin{center}
        Gymnázium \\
        Veľká okružná 22, 010 01 Žilina

        \vspace{7cm}
        \Huge Botanica - simulátor rastlín na báze celulárnych automatónov

        \vspace{1.13cm}
        \Large Stredoškolská odborná činnosť

        \vspace{2.12cm}
        \normalsize Č. odboru: 11
    \end{center}

    \vfill

    \begin{minipage}{0.75\textwidth}
        Riešitelia: František Knapec, Michael Sklenka, Marek Beňo \par
        Ročník štúdia: 4. \par
        Školiteľ: PaedDr. Jana Pekárová, PhD.\par
        Konzultant: Ing. Tomáš Milet, PhD.
    \end{minipage}
    \hfill
    \begin{minipage}{0.23\textwidth}
        \hfil % basically align right
        \begin{tabular}{rc}
            \\ \\
            Mesto: & Žilina \\
            Rok:   & 2025
        \end{tabular}
    \end{minipage}
\end{titlepage}

% zapocitaj strany
\setcounter{page}{3}

%
% Čestné vyhlásenie
%

\thispagestyle{empty}

\noindent
\textbf{Čestné vyhlásenie:}

\noindent
Prehlasujeme, že sme prácu na tému
Botanica - simulátor rastlín na báze celulárnych automatónov
vypracovali samostatne s~použitím literatúry uvedenej v~zozname použitej literatúry.
Zároveň prehlasujeme, že sme predloženú písomnú prácu neprihlásili a ani neprezentovali
v žiadnej inej súťaži, ktorá je pod gestorstvom MŠMVVaŠ SR. Sme si vedomí zákonných dôsledkov,
ak v~nej uvedené údaje nie sú pravdivé.

\newpage


%
% Poďakovanie (nepovinné)
% TODO: chceme?
%

\thispagestyle{empty}

\null
\vfill

\noindent
\textbf{Poďakovanie}

\noindent
Toto je poďakovanie

\vspace{8cm}
\newpage


%
% [x] Obsah
% [ ] Zoznam skratiek, značiek a symbolov (nepovinné)
% [ ] Zoznam tabuliek, grafov a ilustrácií (nepovinné)
%

\thispagestyle{empty}
\tableofcontents
\newpage


%
% The work :)
%

\addcontentsline{toc}{section}{Úvod}
\section*{Úvod}

\section{Problematika a prehľad literatúry}
\section{Ciele práce}
\section{Materiál a metodika}
\section{Výsledky práce}
\section{Diskusia}
\section{Závery práce}
\section{Zhrnutie}
\section{Zoznam použitej literatúry}
% Prílohy (nepovinné)

\end{document}